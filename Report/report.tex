\documentclass[12pt, bibliography=totoc, listof=totoc, svgnames]{scrreprt} 	% Beginn Dokument
\usepackage{cite}				% Fürs Zitieren
\usepackage[utf8]{inputenc} 	% Verwendete ASCII Codierung für ä ö ü
\usepackage[dvips]{graphicx}	% Für Graphen
\usepackage[english]{babel} 		% Englisch als Sprache
\selectlanguage{english}
\usepackage{amssymb}
\usepackage{listings}
\usepackage{color}
\usepackage{gensymb}
\definecolor{mygreen}{rgb}{0.8,0.8,0.8}
\lstset{backgroundcolor=\color{mygreen}}

\usepackage[footnote]{acronym}
%\renewcommand*\bflabel[1]{{\textbf{\textsf{#1}}\dotfill}}
%\renewcommand*{\acsfont}[1]{{\color{black}\textit{#1}}}

\usepackage{todonotes}

%%% Papierformat %%%
\usepackage{geometry}			
\geometry{a4paper}
\geometry{paper=a4paper,left=30 mm,right=30 mm,top=30 mm}

%%% Packages für den Graphen %%%
\usepackage{pgf}
\usepackage{tikz}
\usetikzlibrary{arrows,automata}
\setlength{\parindent}{0mm}

%%% Kopf & Fußzeilen %%%
\usepackage{fancyhdr} % This should be set AFTER setting up the page geometry
\pagestyle{fancy} % options: empty , plain , fancy
\fancypagestyle{plain}{}% damit auch "plain" Seiten fancy werden
\renewcommand{\headrulewidth}{1pt} % customise the layout…
\lhead{\textsc{Fault Detection and Isolation with Sliding Mode Observers}\\
%\emph{\leftmark}
}\chead{}\rhead{}
\lfoot{}\cfoot{}\rfoot{\thepage}
\setlength{\headheight}{40pt}
\renewcommand*{\chapterheadstartvskip}{\vspace*{.1\baselineskip}}% Abstand einstellen

%%% HRule erzeugt eine Linie nach einem Absatz %%%
\newcommand{\HRule}{\rule{\linewidth}{0.5mm}}

%%% Standard Titelblatt Einstellungen %%%
%\title{Bachelorarbeit}
%\author{\small{Sevim Eda}}
%\date{\today}% 

\usepackage{tabularx}
\usepackage{amsmath}

\usepackage{xcolor}
\usepackage{verbatim}

\bibliographystyle{unsrt}

\begin{document}
%\maketitle

\begin{titlepage}

\begin{center}


% Upper part of the page
	%\includegraphics[width=0.60\textwidth]{pics/FH}\\[1cm]    
	\begin{tikzpicture}[font=\sffamily]
	\node [circle] (dot1) {};
	\node [circle] (dot2) at ($(dot1.west) + (-0.4, 0)$) {};
	\node [circle] (dot3) at ($(dot1.east) + (+0.4, 0)$) {};
	\node [circle] (dot4) at ($(dot2.north) + (0, 0.4)$) {};
	\node [circle] (dot5) at ($(dot3.north) + (0, 0.4)$) {};
	\node [circle] (dot6) at ($(dot2.south) + (0, -0.4)$) {};
	\node [circle] (dot7) at ($(dot3.south) + (0, -0.4)$) {};
	
	\node [circle] (fdot1) at ($(dot1) + (-2, 0)$) {};
	\node [circle] (fdot2) at ($(fdot1.west) + (-0.4, 0)$) {};
	\node [circle] (fdot3) at ($(fdot1.north) + (0, 0.4)$) {};
	\node [circle] (fdot4) at ($(fdot2.south) + (0, -0.4)$) {};
	\node [circle] (fdot5) at ($(fdot2.north) + (0, 0.4)$) {};
	\node [circle] (fdot6) at ($(fdot3.east) + (0.4, 0)$) {};
	
	\draw [ultra thick] ($(fdot5.north west)+(-0.5, 0.6)$) -- ($(fdot4.south west)+(-0.5, -0.6)$);
	
	\node [red!80!black, text width=2.6cm, align=flush center] (text1) at ($(fdot1) + (-2.9, 0.6)$) {\textbf{C\hfill A\hfill R\hfill I\hfill N\hfill T\hfill H\hfill I\hfill A}};
	\node [text width=2.6cm, align=center] (text2) at ($(text1.south) + (0, -0.13)$) {\textbf{U\hfill N\hfill I\hfill V\hfill E\hfill R\hfill S\hfill I\hfill T\hfill Y}};
	\node [text width=2.6cm, align=center] (text3) at ($(text2.south) + (0, -0.13)$) {\textbf{O\hfill F \hfill A\hfill P\hfill P\hfill L\hfill I\hfill E\hfill D}};
	\node [text width=2.6cm, align=center] (text4) at ($(text3.south) + (0, -0.13)$) {\textbf{S\hfill C\hfill I\hfill E\hfill N\hfill C\hfill E\hfill S}};
	
	\node [text width=5cm, align=center] (text5) at ($(dot4.west) + (2.5, 0.6)$) {\textbf{F\hfill A\hfill C\hfill H\hfill H\hfill O\hfill C\hfill H\hfill S\hfill C\hfill H\hfill U\hfill L\hfill E}};
	\node [red!80!black, text width=3.1cm, font={\Large\sffamily}] (text6) at ($(dot5.east) + (1.92, 0.05)$) {\textbf{K\hfill Ä\hfill R\hfill N\hfill T\hfill E\hfill N}};
	
	\fill (dot1) circle [radius=7pt];
	\fill (dot2) circle [radius=7pt];
	\fill (dot3) circle [radius=7pt];
	\fill (dot4) circle [radius=7pt];
	\fill (dot5) circle [radius=7pt];
	\fill (dot6) circle [radius=7pt];
	\fill (dot7) circle [radius=7pt];
	
	\fill [red!80!black] (fdot1) circle [radius=7pt];
	\fill [red!80!black] (fdot2) circle [radius=7pt];
	\fill [red!80!black] (fdot3) circle [radius=7pt];
	\fill [red!80!black] (fdot4) circle [radius=7pt];
	\fill [red!80!black] (fdot5) circle [radius=7pt];
	\fill [red!80!black] (fdot6) circle [radius=7pt];
\end{tikzpicture}\\[1cm]
	
\textsc{\LARGE Carinthia University of Applied Sciences}\\[1.5cm]

\textsc{\Large Degree Program: Systems Design}\\[5cm]


% Title
\HRule \\[1cm]
{ \huge \bfseries EXTRA QUESTIONS }\\[.5cm]
"\course"\\[.3cm]
\labtitle

\HRule \\[3cm]



\vfill

	\begin{table}[H]
		\centering
		\begin{tabular}{|l|c|}
			\toprule
			Student: & \stusurname\ \textsc{\stuname} \\
			ID: & \stuid   	\\
			\midrule
			Lecturer: & \lecturer  \\
			Submitted: & {\large \today}  	\\
			Submitted: &  \\
			Grade: &  \\
			\bottomrule
		\end{tabular}
	\end{table}


\end{center}

\end{titlepage}
%\tableofcontents
%\renewcommand{\thepage}{\Roman{page}} \setcounter{page}{1}
%\include{abstract}
%\listoffigures
%\listoftables
%\include{acronym}
%\renewcommand{\chaptermark}[1]{\markboth{\thechapter.\ #1}{}}
\renewcommand{\thepage}{\arabic{page}} \setcounter{page}{1}

\section*{Description of the research issue}
In today's industry there is a clear trend into actively controlling more and more parameters of industrial machines. The advantages of the increasing number of controlled parameters include increased performance, flexibility, better environmental compatibility and many more.  \\ 
Although the overall performance of an industrial machine can be increased with advanced controlling techniques, the process within those machines become more prone to errors and malfunctions. To avoid excessive downtimes of the machines or faults in the production it is important to monitor the processes within the machines and detect errors and malfunctions in an early state.\\
Since monitoring of the processes with dedicated sensor systems can be expensive and difficult to implement, there has been methods developed which are able to identify errors and malfunctions based on sensor devices already needed for the implemented controllers. These methods can be summed up under the term "Fault Detections and Isolation". Most of these approaches are based on the models and the mathematical descriptions of the systems, which should be monitored. Since those models are always not a prefect description of the real world system, the applied methods should be robust to this variations within the system.\\
The   

\section*{Main goal and scope of this work}
The aim of this thesis is to analyze a system of heater and a wafer, to quantify the coupling between the different heating zones and implement actions to reduce the coupling in between. The final target is to achieve an homogeneous temperature profile of across the entire wafer.
 
\section*{Methods and research approach}
As one of the first steps the system must be identified. This happens only with experimental data. With the results from the system-identification the coupling between the individual systems can be quantified. After the coupling of the systems has been calculated, the values are analyzed and an overall evaluation of the systems is performed.

\section*{Course of actions}

\section*{Thesis content}
%\include{Theory}
%\include{Identification}
%\include{Analysis}
%\include{Conclusion}
%\bibliography{bibliography}
%\include{Statement}

%%% Neues Kapitel %%%
%\chapter{Kapitel}
%\label{sec:Kapitel}
%
%\section{Unterpunkt}

%%% Bild einfügen (nur .eps) %%%
%\begin{figure}[h]
%	\centering
%	\includegraphics[width=0.85\textwidth]{pics/xyz}\\
%	\caption{Parameter Sweep für die Weite des NMOS4}
%	\label{fig:xyz}
%\end{figure}
% Simulink_Modell_Prints('name_simulink_datei','p')

%%% Tabelle einfügen
%\begin{table}[h]
%	\begin{center}
%		\caption{SPICE Parameter für NMOS und PMOS} 
%		\begin{tabular}{|l|l|l|}
%			\hline
%			x & y & z\\
%			\hline
%		\end{tabular}
%		\label{tab:}
%	\end{center}
%\end{table}

%%% Formel einfügen
%\begin{equation}
%	\centering
%	Formel
%	\label{eq:}
%\end{equation}

\end{document}
